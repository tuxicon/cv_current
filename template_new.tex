%% start of file `template.tex'.
%% Copyright 2006-2012 Xavier Danaux (xdanaux@gmail.com).
%
% This work may be distributed and/or modified under the
% conditions of the LaTeX Project Public License version 1.3c,
% available at http://www.latex-project.org/lppl/.



\documentclass[11pt,a4paper,sans]{moderncv}   % possible options include font size ('10pt', '11pt' and '12pt'), paper size ('a4paper', 'letterpaper', 'a5paper', 'legalpaper', 'executivepaper' and 'landscape') and font family ('sans' and 'roman')

% moderncv themes
\moderncvstyle{classic}                        % style options are 'casual' (default), 'classic', 'oldstyle' and 'banking'
\moderncvcolor{blue}                          % color options 'blue' (default), 'orange', 'green', 'red', 'purple', 'grey' and 'black'
%\renewcommand{\familydefault}{\sfdefault}    % to set the default font; use '\sfdefault' for the default sans serif font, '\rmdefault' for the default roman one, or any tex font name
%\nopagenumbers{}                             % uncomment to suppress automatic page numbering for CVs longer than one page

% character encoding
%\usepackage[utf8]{inputenc}                  % if you are not using xelatex ou lualatex, replace by the encoding you are using
%\usepackage{CJKutf8}                         % if you need to use CJK to typeset your resume in Chinese, Japanese or Korean
\newcommand\Colorhref[3][cyan]{\href{#2}{\small\color{#1}#3}}
% adjust the page margins
\usepackage[scale=0.85]{geometry}
%\setlength{\hintscolumnwidth}{3cm}           % if you want to change the width of the column with the dates
%\setlength{\makecvtitlenamewidth}{10cm}      % for the 'classic' style, if you want to force the width allocated to your name and avoid line breaks. be careful though, the length is normally calculated to avoid any overlap with your personal info; use this at your own typographical risks...

% personal data
\setlength{\makecvtitlenamewidth}{12cm}
\renewcommand*{\namefont}{\fontsize{24}{29}\mdseries\upshape}

\firstname{Saiganesh}
\familyname{Swaminathan}
\title{Masters Student}                          % optional, remove / comment the line if not wanted
\address{107 Boulevard de Mond\'etour}{91400 Orsay, France}    % optional, remove / comment the line if not wanted
\mobile{+33658513433}                     % optional, remove / comment the line if not wanted
  % optional, remove / comment the line if not wanted
\homepage{www.saiganesh.net}                    % optional, remove / comment the line if not wanted
\extrainfo{\emailsymbol\emaillink{saiganesh.swaminathan@masterschool.eitictlabs.eu}}           % optional, remove / comment the line if not wanted
              % optional, remove / comment the line if not wanted; '64pt' is the height the picture must be resized to, 0.4pt is the thickness of the frame around it (put it to 0pt for no frame) and 'picture' is the name of the picture file
%\quote{Some quote}                            % optional, remove / comment the line if not wanted

% to show numerical labels in the bibliography (default is to show no labels); only useful if you make citations in your resume
%\makeatletter
%\renewcommand*{\bibliographyitemlabel}{\@biblabel{\arabic{enumiv}}}
%\makeatother
%\renewcommand*{\bibliographyitemlabel}{[\arabic{enumiv}]% CONSIDER REPLACING THE ABOVE BY THIS



\makeatletter
\renewcommand*{\makecvtitle}{%
  % recompute lengths (in case we are switching from letter to resume, or vice versa)
  \recomputecvlengths%
  % optional detailed information box
  \newbox{\makecvtitledetailsbox}%
  \savebox{\makecvtitledetailsbox}{%
    \addressfont\color{color2}%
    \begin{tabular}[t]{@{}r@{}}%
      \ifthenelse{\isundefined{\@addressstreet}}{}{\makenewline\addresssymbol\@addressstreet%
        \ifthenelse{\equal{\@addresscity}{}}{}{\makenewline\@addresscity}}% if \addresstreet is defined, \addresscity will always be defined but could be empty
      \ifthenelse{\isundefined{\@mobile}}{}{\makenewline\mobilesymbol\@mobile}%
      \ifthenelse{\isundefined{\@email}}{}{\makenewline\emailsymbol\emaillink{\@email}}%
      \ifthenelse{\isundefined{\@homepage}}{}{\makenewline\homepagesymbol\httplink{\@homepage}}%
      \ifthenelse{\isundefined{\@extrainfo}}{}{\makenewline\@extrainfo}%
    \end{tabular}
  }%
  % optional photo (pre-rendering)
  \newbox{\makecvtitlepicturebox}%
  \savebox{\makecvtitlepicturebox}{%
    \ifthenelse{\isundefined{\@photo}}%
    {}%
    {%
      \hspace*{\separatorcolumnwidth}%
      \color{color1}%
      \setlength{\fboxrule}{\@photoframewidth}%
      \ifdim\@photoframewidth=0pt%
        \setlength{\fboxsep}{0pt}\fi%
  \framebox{\includegraphics[width=\@photowidth]{\@photo}}}}%
  % name and title
  \newlength{\makecvtitledetailswidth}\settowidth{\makecvtitledetailswidth}{\usebox{\makecvtitledetailsbox}}%
  \newlength{\makecvtitlepicturewidth}\settowidth{\makecvtitlepicturewidth}{\usebox{\makecvtitlepicturebox}}%
  \ifthenelse{\lengthtest{\makecvtitlenamewidth=0pt}}% check for dummy value (equivalent to \ifdim\makecvtitlenamewidth=0pt)
    {\setlength{\makecvtitlenamewidth}{\textwidth-\makecvtitledetailswidth-\makecvtitlepicturewidth}}%
    {}%
  \begin{minipage}[t]{\makecvtitlenamewidth}%
    \namestyle{\@firstname\ \@familyname}%
    \ifthenelse{\equal{\@title}{}}{}{\\[1.25em]\titlestyle{\@title}}%
  \end{minipage}%
  \hfill%
  % detailed information
  \llap{%
    \begin{minipage}[t]{\makecvtitledetailswidth}%
    \vspace*{-17pt}%
    \usebox{\makecvtitledetailsbox}%
    \end{minipage}}% \llap is used to suppress the width of the box, allowing overlap if the value of makecvtitlenamewidth is forced
  % optional photo (rendering)
  \begin{minipage}[t]{\makecvtitlepicturewidth}%
    \vspace*{-17pt}%
    \vbox to 0pt{%
      \usebox{\makecvtitlepicturebox}%
    }%
  \end{minipage}\\[2.5em]%
  % optional quote
  \ifthenelse{\isundefined{\@quote}}%
    {}%
    {{\centering\begin{minipage}{\quotewidth}\centering\quotestyle{\@quote}\end{minipage}\\[2.5em]}}%
  \par}% to avoid weird spacing bug at the first section if no blank line is left after \makecvtitle
\makeatother



% bibliography with mutiple entries
%\usepackage{multibib}
%\newcites{book,misc}{{Books},{Others}}
%----------------------------------------------------------------------------------
%            content
%----------------------------------------------------------------------------------
\begin{document}
%\begin{CJK*}{UTF8}{gbsn}                     % to typeset your resume in Chinese using CJK
%-----       resume       ---------------------------------------------------------
\makecvtitle



\section{Research Interests}

\cvitem{Summary} {I'm excited to be a part of research ventures that lead to building interactive systems, toolkits and tools that leverages the existing theories on human capabilities, interaction contexts in real world and novel device modalities} 

\cvitem{Interested Areas} {Interaction techniques, Design of Interactive systems, Methods and theroies for understanding users} 


\section{Education}

\cventry{2012--2014}{Masters in Human Computer Interaction and Design} {Universit\`e Paris-Sud}{Paris}{France}{ Dual degree masters program with TU Berlin acting as the second university with the specialization of multi-modal interaction. The masters program is organised by {\textbf{\textsf{EIT ICT Labs Masters School}}} - European Institute of Technology.\\ 
\textbf{\textsf{Relevant Courses}}: Introduction to HCI, Programming Interactive Systems, User-centered design methods, Interactive computer graphics}
\medskip
\cventry{2008--2012}{B.Tech in Computer Science and Engineering}{Shanmuga Arts Science Technology and Research Academy}{Thanjavur, TN, India}{\textit{GPA: 8/10}}{\textbf{\textsf{Relevant Courses}}: Pervasive Computing, Design and Analysis of Algorithms, Theory Of Computation, Object Oriented Analysis and Design}
\medskip  % arguments 3 to 6 can be left empty
\cventry{2011--2012}{Exchange Student at department of computer science}{ETH Z\"urich}{ Z\"urich, Switzerland}{\textit{4.95/6}}{\textbf{\textsf{Relevant Courses}}: Bachelor-Thesis, Human Computer Interaction, Research in Computer Science, Web Engineering, Software Engineering Laboratory, Software Architecture}

\section{Publications}
\begin{itemize}
  \item  Recent work on physical visualizations tool submitted to CHI 2014 for review (manuscript available after review)
  
  \item Saiganesh Swaminathan, PIM Touch: Extending Personal Information Management Paradigms for Multi-touch Interaction Contexts, \emph{Bachelor Thesis}, ETH Z\"urich, Feb 2012 
  
   
\end{itemize}  

\section{Scholarships}
			\begin{itemize}
			  \item \textbf{\textsf{EIT ICT Labs Excellence Nominee}} scholarship, tuition fee waiver and travel support for attending two graduate schools. Awarded by European Insitute of Technology for entire duration of the masters program. 
				\medskip
				\item \textbf{\textsf{Desh-Videsh}} scholarship for pursuing academic endeavours abroad awarded by SASTRA university which covered round trip travel and living expenses.
				\medskip
				\item Schloarship awarded by \textbf{\textsf{Global information systems group, ETH Z\"urich}} to pursue research activities at ETH Z\"urich
			\end{itemize}

\section{Selected Research and Project Experiences}


\cvitem{Project} {\textbf{\textsf{Toolkit for building physical visualizations}}}
\cventry{May 2013--Sep 2013}{\textnormal {\em{Research, Intern}}}{AVIZ group at INRIA-Saclay}{Advisors: Dr. Pierre Dragicevic, Yvonne Jansen}{}{ My role in project involved designing a prototype to help end users build \Colorhref{http://www.aviz.fr/Research/Phys}{Physical Visualizations}. The tool that we designed is capable of generating parameterizable physical visualizations through the generation of design files which can be fabricated with laser cutters, 3D printers, etc. Further more we conducted design sessions with users and results were submitted as a paper to CHI 2014} 

\cvitem{Project} {\textbf{\textsf{FloorCom: Interactive floor comunication for conference attendees}}}
\cventry{Jan 2013--Mar 2013}{\textnormal {\em{Student, CourseProject}}}{Universit\`e Paris-Sud}{Professor: Dr. Wendy Mackay}{}{ As a part of the user centered design course we developed a video prototype of a concept called \Colorhref{https://docs.google.com/file/d/0B0wU0tKL7ixrWi1uRHZhUUFYZDQ/edit?usp=sharing}{FloorCom}. As a part of the course we did Interviews, used grounded theory, created user personas, profiles, design scenarios, design space and storyboards. The design scenarios were further refined and redesigned with theories from social sciences and finally leading to a video prototype.} 



\cvitem{Project} {\textbf{\textsf{TweetZoom: An Exploration of Interaction Technique}}}
\cventry{Sep 2012--Dec 2012}{\textnormal {\em{Student, CourseProject}}}{Universit\`e Paris-Sud}{Professor: Dr. Michel Beaudouin-Lafon}{}{ We explored a new interaction technique (\Colorhref{https://engineering.purdue.edu/~elm/projects/polyzoom/polyzoom.pdf}{PolyZoom}) from CHI 2012 by implementing a mash-up application which uses the technique. The application builds on Twitter search API and Google maps API. It allows user to navigate tweets around the world at different scales of zoom and spatially distant parts of the world simultaneously. 
}


\cvitem{Project} {\textbf{\textsf{Semester Project: Computer Vision Labratory}}}
\cventry{Oct 2011--Nov 2011}{\textnormal {\em{Student, CourseProject}}}{ETH Z\"urich}{Advisors: Dr. Helmut Grabner, Prof. Luc Van Gool}{}{ Worked on a project to improve the machine learning algorithm that detects and categorizes objects based on affordances of objects. We  investigated various mechanisms through which the algorithm could be improved for increased scientific result like physical stability of objects, material Properties in 3d content scene discovery.\newline
}

\cvitem{Project} {\textbf{\textsf{Integrating Multi-Touch Interactions in Existing Web Interfaces}}}
\cventry{Jun 2011--Sep 2011}{\textnormal {\em{Research Intern}}}{Global Information Systems group, ETH Z\"urich}{Advisors: Dr. Michael Nebeling, Prof. Moira Norrie}{}{We investigated the gradual adaptation of existing web interfaces to touch and multi-touch devices using \Colorhref{http://dev.globis.ethz.ch/jqmultitouch/js/jquery.multitouch.js}{jQMultiTouch}. jQMultiTouch is a javascript based framework for developing multi-touch interactions on the web. The project began by developing several multi-touch prototypes of existing web interface picnik (Photo editing tool from Flickr). The motivation for these prototypes are both evaluating the framework and the interactions of the newly adapted website. The work ended with designing user-studies for comparing the existing web interfaces with the adapted interfaces. \newline{}%
}

\cvitem{Project} {\textbf{\textsf{libDB: API for relational database access from eiffel}}}
\cventry{ Feb 2011--May 2011}{\textnormal {\em{Student, CourseProject}}}{ETH Z\"urich}{Professor: Bertrand Meyer}{}{ We conceptualized, designed and implemented an RDB API for eiffel language. We followed a waterfall software development cycle through the design of the architecture (of API). The various design decisions included contracts, design patterns, various Public/Private classes with their methods. The project was a team based project for the course software architecture, It was international with four developers from four different countries \newline{}%
}


\cvitem{Project} {\textbf{\textsf{libSpell: Cross platform eiffel spelling library}}}
\cventry{ Mar 2011--May 2011}{\textnormal {\em{Student, CourseProject}}}{ETH Z\"urich}{Professor: Bertrand Meyer}{}{ Developed a cross platform eiffel library that uses google toolbar api to check spelling and hint programmers for suggestions. The API was developed as a part of Software Engineering laboratory course}

\subsection{Miscellaneous}

\cventry{May 2010--Dec 2010}{Student SysAdmin}{SASTRA University}{Thanjavur, India}{}{ Responsibilities included 1) Implementing university wide wi-fi authentication/access mechanisms 2)Implementing university wide LDAP server providing the provision of ldap-squid authentication mechanism 3)Integrating the university educational system MOODLE with ldap configurations}

\cventry{ June 2009-- May 2010}{Fedora Ambassador}{Fedora}{Thanjavur District}{}{ Was involved in organizing hackathons, teaching and evangelizing the use of various open source technologies around the region. Roles also included several development decisions on features of fedora over IRC}


\section{Skill Set}
\cvdoubleitem{HCI Skills:}{Paper Prototyping, Semi-structured
interviews, Critical Incident Technique, Video Prototyping, Study Design}{Ides and software:}{Eclipse, Vim, NetBeans, Familiar with UNIX/Linux environments, Revision control (Mercurial, SVN, Git), Unit testing, and Code review systems}
\cvdoubleitem {Programming Languages:}{C, C++, Java (J2EE, Swing), Eiffel}{Sysadmin skills:}{Apache, Memcached, Squid, NFS, DHCP, NTP, SSH, DNSandSNMP}
\cvdoubleitem{Web Technologies:}{PHP, HTML5, CSS, JavaScript, JavaFX, jQuery, AJAX, NodeJS}{Other Tools:}{R statistical modelling(Moderate),\LaTeX\ }
\cvdoubleitem{DBMS: }{MySQL, SQLlite, Oracle, SQL Server}{Other:}{Experience in academic research, experimental design and methodology}


\section{Languages}
\cvitemwithcomment{English}{Fluent}{}
\cvitemwithcomment{German}{Intermediate}{}
\cvitemwithcomment{Tamil}{Fluent}{}
\cvitemwithcomment{French}{Basic}{}

\section{Clubs and Certifications}
\cvlistitem{An member of \textbf{\textsf{GLOSS}} \emph{(GNU Linux \& Open Source Society)} at SASTRA. Roles include coordinating and conducting sessions on various fields in FOSS, including Linux customizations and hackathons}
\cvlistitem{RedHat certified engineer}
\cvlistitem{RedHat certified expertise in Directory Services}
\cvlistitem{RedHat certified expertise in Administration of SeLinux}


\section{References}

Available upon request.









%\clearpage\end{CJK*}                         % if you are typesetting your resume in Chinese using CJK; the \clearpage is required for fancyhdr to work correctly with CJK, though it kills the page numbering by making \lastpage undefined
\end{document}


%% end of file `template.tex'.


