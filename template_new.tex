%% start of file `template.tex'.
%% Copyright 2006-2012 Xavier Danaux (xdanaux@gmail.com).
%
% This work may be distributed and/or modified under the
% conditions of the LaTeX Project Public License version 1.3c,
% available at http://www.latex-project.org/lppl/.



\documentclass[11pt,a4paper,sans]{moderncv}   % possible options include font size ('10pt', '11pt' and '12pt'), paper size ('a4paper', 'letterpaper', 'a5paper', 'legalpaper', 'executivepaper' and 'landscape') and font family ('sans' and 'roman')

% moderncv themes
\moderncvstyle{classic}                        % style options are 'casual' (default), 'classic', 'oldstyle' and 'banking'
\moderncvcolor{blue}                          % color options 'blue' (default), 'orange', 'green', 'red', 'purple', 'grey' and 'black'
%\renewcommand{\familydefault}{\sfdefault}    % to set the default font; use '\sfdefault' for the default sans serif font, '\rmdefault' for the default roman one, or any tex font name
%\nopagenumbers{}                             % uncomment to suppress automatic page numbering for CVs longer than one page

% character encoding
%\usepackage[utf8]{inputenc}                  % if you are not using xelatex ou lualatex, replace by the encoding you are using
%\usepackage{CJKutf8}                         % if you need to use CJK to typeset your resume in Chinese, Japanese or Korean
\newcommand\Colorhref[3][cyan]{\href{#2}{\small\color{#1}#3}}
% adjust the page margins
\usepackage[scale=0.85]{geometry}
%\setlength{\hintscolumnwidth}{3cm}           % if you want to change the width of the column with the dates
%\setlength{\makecvtitlenamewidth}{10cm}      % for the 'classic' style, if you want to force the width allocated to your name and avoid line breaks. be careful though, the length is normally calculated to avoid any overlap with your personal info; use this at your own typographical risks...

% personal data
\setlength{\makecvtitlenamewidth}{12cm}
\renewcommand*{\namefont}{\fontsize{24}{29}\mdseries\upshape}

\firstname{Saiganesh}
\familyname{Swaminathan}
\title{Masters Student}                          % optional, remove / comment the line if not wanted
\address{Xerox Research Centre Europe \\ 6 chemin de Maupertuis\\}{Meylan, France, 38240}    % optional, remove / comment the line if not wanted
\mobile{+33663028548}                     % optional, remove / comment the line if not wanted
  % optional, remove / comment the line if not wanted
\homepage{www.saiganesh.net}                    % optional, remove / comment the line if not wanted
\extrainfo{\emailsymbol\emaillink{saiganesh.swaminathan@masterschool.eitictlabs.eu}}           % optional, remove / comment the line if not wanted
              % optional, remove / comment the line if not wanted; '64pt' is the height the picture must be resized to, 0.4pt is the thickness of the frame around it (put it to 0pt for no frame) and 'picture' is the name of the picture file
%\quote{Some quote}                            % optional, remove / comment the line if not wanted

% to show numerical labels in the bibliography (default is to show no labels); only useful if you make citations in your resume
%\makeatletter
%\renewcommand*{\bibliographyitemlabel}{\@biblabel{\arabic{enumiv}}}
%\makeatother
%\renewcommand*{\bibliographyitemlabel}{[\arabic{enumiv}]% CONSIDER REPLACING THE ABOVE BY THIS



\makeatletter
\renewcommand*{\makecvtitle}{%
  % recompute lengths (in case we are switching from letter to resume, or vice versa)
  \recomputecvlengths%
  % optional detailed information box
  \newbox{\makecvtitledetailsbox}%
  \savebox{\makecvtitledetailsbox}{%
    \addressfont\color{color2}%
    \begin{tabular}[t]{@{}r@{}}%
      \ifthenelse{\isundefined{\@addressstreet}}{}{\makenewline\addresssymbol\@addressstreet%
        \ifthenelse{\equal{\@addresscity}{}}{}{\makenewline\@addresscity}}% if \addresstreet is defined, \addresscity will always be defined but could be empty
      \ifthenelse{\isundefined{\@mobile}}{}{\makenewline\mobilesymbol\@mobile}%
      \ifthenelse{\isundefined{\@email}}{}{\makenewline\emailsymbol\emaillink{\@email}}%
      \ifthenelse{\isundefined{\@homepage}}{}{\makenewline\homepagesymbol\httplink{\@homepage}}%
      \ifthenelse{\isundefined{\@extrainfo}}{}{\makenewline\@extrainfo}%
    \end{tabular}
  }%
  % optional photo (pre-rendering)
  \newbox{\makecvtitlepicturebox}%
  \savebox{\makecvtitlepicturebox}{%
    \ifthenelse{\isundefined{\@photo}}%
    {}%
    {%
      \hspace*{\separatorcolumnwidth}%
      \color{color1}%
      \setlength{\fboxrule}{\@photoframewidth}%
      \ifdim\@photoframewidth=0pt%
        \setlength{\fboxsep}{0pt}\fi%
  \framebox{\includegraphics[width=\@photowidth]{\@photo}}}}%
  % name and title
  \newlength{\makecvtitledetailswidth}\settowidth{\makecvtitledetailswidth}{\usebox{\makecvtitledetailsbox}}%
  \newlength{\makecvtitlepicturewidth}\settowidth{\makecvtitlepicturewidth}{\usebox{\makecvtitlepicturebox}}%
  \ifthenelse{\lengthtest{\makecvtitlenamewidth=0pt}}% check for dummy value (equivalent to \ifdim\makecvtitlenamewidth=0pt)
    {\setlength{\makecvtitlenamewidth}{\textwidth-\makecvtitledetailswidth-\makecvtitlepicturewidth}}%
    {}%
  \begin{minipage}[t]{\makecvtitlenamewidth}%
    \namestyle{\@firstname\ \@familyname}%
    \ifthenelse{\equal{\@title}{}}{}{\\[1.25em]\titlestyle{\@title}}%
  \end{minipage}%
  \hfill%
  % detailed information
  \llap{%
    \begin{minipage}[t]{\makecvtitledetailswidth}%
    \vspace*{-17pt}%
    \usebox{\makecvtitledetailsbox}%
    \end{minipage}}% \llap is used to suppress the width of the box, allowing overlap if the value of makecvtitlenamewidth is forced
  % optional photo (rendering)
  \begin{minipage}[t]{\makecvtitlepicturewidth}%
    \vspace*{-17pt}%
    \vbox to 0pt{%
      \usebox{\makecvtitlepicturebox}%
    }%
  \end{minipage}\\[2.5em]%
  % optional quote
  \ifthenelse{\isundefined{\@quote}}%
    {}%
    {{\centering\begin{minipage}{\quotewidth}\centering\quotestyle{\@quote}\end{minipage}\\[2.5em]}}%
  \par}% to avoid weird spacing bug at the first section if no blank line is left after \makecvtitle
\makeatother



% bibliography with mutiple entries
%\usepackage{multibib}
%\newcites{book,misc}{{Books},{Others}}
%----------------------------------------------------------------------------------
%            content
%----------------------------------------------------------------------------------
\begin{document}
%\begin{CJK*}{UTF8}{gbsn}                     % to typeset your resume in Chinese using CJK
%-----       resume       ---------------------------------------------------------
\makecvtitle



\section{Research Interests}

\cvitem{Summary} {I'm excited to be a part of research ventures that lead to building interactive systems, toolkits and tools that leverages the existing theories on human capabilities, interaction contexts in real world and novel device modalities} 

\cvitem{Interested Areas} {Interaction techniques, Design of Interactive systems, Methods and theroies for understanding users} 


\section{Education}

\cventry{2012--2014}{Masters in Human Computer Interaction and Design} {Universit\`e Paris-Sud}{Paris}{France}{ Dual degree masters program with TU Berlin as the second university. The masters program is organised by {\textbf{\textsf{EIT ICT Labs Masters School}}} - European Institute of Technology.\\ 
\textbf{\textsf{Relevant Courses}}: Introduction to HCI, Programming Interactive Systems, User-centered design methods, Interactive computer graphics}
\medskip
\cventry{2008--2012}{B.Tech in Computer Science and Engineering}{Shanmuga Arts Science Technology and Research Academy}{Thanjavur, TN, India}{\textit{GPA: 8/10}}{\textbf{\textsf{Relevant Courses}}: Pervasive Computing, Design and Analysis of Algorithms, Theory Of Computation, Object Oriented Analysis and Design}
\medskip  % arguments 3 to 6 can be left empty
\cventry{2011--2012}{Exchange Student at department of computer science}{ETH Z\"urich}{ Z\"urich, Switzerland}{\textit{4.95/6}}{\textbf{\textsf{Relevant Courses}}: Bachelor-Thesis, Human Computer Interaction, Research in Computer Science, Web Engineering, Software Engineering Laboratory, Software Architecture}

\section{Publications}
\begin{itemize}
  \item  \textbf{Saiganesh Swaminathan}, Conglei Shi, Yvonne Jansen, Pierre Dragicevic, Lora Oehlberg, Jean-Daniel Fekete. Supporting The Design and Fabrication of Physical Visualizations. \textit{CHI 2014}: Proceedings of the SIGCHI Conference on Human Factors in Computing Systems, ACM, pages 3845-3854, April 2014.\\.
  
  \item   \textbf{Saiganesh Swaminathan}, Conglei Shi, Yvonne Jansen, Pierre Dragicevic, Lora Oehlberg, Jean-Daniel Fekete. Creating Physical Visualizations With MakerVis. Interactivity Demo at \textit{CHI 2014}: Extended Abstracts of the SIGCHI Conference on Human Factors in Computing Systems, ACM, pages 543-546, April 2014. \\.
  
  \item \textbf{Saiganesh Swaminathan}, PIM Touch: Extending Personal Information Management Paradigms for Multi-touch Interaction Contexts, \emph{Bachelor Thesis}, ETH Z\"urich, Feb 2012 
  
   
\end{itemize}

\section{Patents}

Supporting skill development for workers through a fair, efficient scheduling of crowdsourcing tasks (Under Filing) Ref No: 20140468US01, ID: 87846077  

\section{Selected Research Experiences}

\cvitem{Project} {\textbf{\textsf{TurkBench: Supporting crowdworkers invisible challenges to "turking"}}}
\cventry{\scriptsize{April 2014--July 2014}}{\textnormal {\em{Researh Intern}}}{Work Practice Technology group at \textbf{Xerox Research Europe Centre}}{Advisers: Ben Hanrahan, Dr. David Martin}{}{ This work was done as a part of the course requirement for EIT ICT Labs industry internship. I assisted the work practice team in building an interactive scheduler for scheduling tasks to workers in crowdsourcing environments. The work led to filing of a patent and submission to conferences are in progress.\\}  

\cvitem{Project} {\textbf{\textsf{Supporting design and fabrication of physical visualizations}}}
\cventry{\scriptsize{May 2013--Sep 2013}}{\textnormal {\em{Research Intern}}}{AVIZ group at \textbf{INRIA-Saclay}}{Advisers: Dr. Yvonne Jansen, Dr. Pierre Dragicevic}{}{ We designed a prototype -- \Colorhref{http://www.aviz.fr/makervis}{MakerVis} to help non-experts in fabrication to build \Colorhref{http://www.aviz.fr/Research/Phys}{physical visualizations}. The tool generates parametric design files based on data specific properties and fabrication specific properties. Further more we conducted design sessions with users and results were submitted as a papers to CHI 2014\\} 

\cvitem{Project} {\textbf{\textsf{Integrating and Extending Multi-Touch Interactions context to Existing Applications}}}
\cventry{\scriptsize{Jun 2011--Jan 2012}}{\textnormal {\em{Research Intern \& bachelor thesis student}}}{Global Information Systems group at\textbf{ ETH Z\"urich}}{Advisers: Dr. Michael Nebeling, Prof. Moira Norrie}{}{We investigated the gradual adaptation of existing web applications to touch and multi-touch devices using \Colorhref{http://dev.globis.ethz.ch/jqmultitouch/js/jquery.multitouch.js}{jQMultiTouch}. jQMultiTouch is a javascript based framework for developing multi-touch interactions on the web. My role in the project involved extending the Personal Information Management (PIM) paradigms to multitouch interaction contexts by re-engineering existing PIM applications. Further, I also helped in evaluating the framework by building these applications. The results of the work are published as part of my bachelor thesis. \newline{}%
}

\section{Skill Set}
\cvdoubleitem{HCI Skills:}{Paper Prototyping, Semi-structured
interviews, Critical Incident Technique, User Studies, Video Prototyping, Study Design}{Ides and software:}{Eclipse, Vim, NetBeans, Familiar with UNIX/Linux environments, Revision control (Mercurial, SVN, Git)}
\cvdoubleitem {Programming Languages:}{C, C++, Java (J2EE, Swing), Eiffel}{Sysadmin skills:}{Apache, Squid, NFS, DHCP, NTP, SSH, DNSandSNMP}
\cvdoubleitem{Web Technologies:}{PHP, HTML5, CSS, JavaScript, JavaFX, jQuery, AJAX, NodeJS}{Other Tools:}{R statistical modelling(Moderate),\LaTeX\ }
\cvdoubleitem{DBMS: }{MySQL, SQLlite, Oracle, SQL Server}{Other:}{Experience in academic research, experimental design and methodology}


\section{Scholarships}
			\begin{itemize}
			  \item \textbf{\textsf{EIT ICT Labs Excellence Nominee}} which includes stipend, tuition fee waiver and travel support for attending two graduate schools. Awarded by European Insitute of Technology for entire duration of the masters program\\. 
			
				\item \textbf{\textsf{Desh-Videsh}} scholarship for pursuing academic endeavours abroad awarded by SASTRA university which covered round trip travel and living expenses\\.
			
				\item Schloarship awarded by \textbf{\textsf{Global information systems group, ETH Z\"urich}} to pursue research activities at ETH Z\"urich
			\end{itemize}

\section{Selected Project Experiences}


\cvitem{Project} {\textbf{\textsf{FloorCom: Interactive floor communication for conference attendees}}}
\cventry{\scriptsize{Jan 2013--Mar 2013}}{\textnormal {\em{Student, CourseProject}}}{Universit\`e Paris-Sud}{Professor: Wendy Mackay}{}{ As a part of the user centered design course we developed a video prototype of a concept called \Colorhref{https://docs.google.com/file/d/0B0wU0tKL7ixrWi1uRHZhUUFYZDQ/edit?usp=sharing}{Floorcom}. Floorcom is an interactive floor that helps conference attendees be aware of each others presence and helps in performing conference specific activities such as connecting with researchers, finding schedules for different talks, etc. As a part of the course we learned to do interviews, use grounded theory, create user personas, profiles, design scenarios, design space and storyboards. The design scenarios were further refined and redesigned with theories from social sciences and finally leading to a video prototype.} 



\cvitem{Project} {\textbf{\textsf{TweetZoom: Exploration of Twitter with PolyZoom}}}
\cventry{\scriptsize{Sep 2012--Dec 2012}}{\textnormal {\em{Student, CourseProject}}}{Universit\`e Paris-Sud}{Professor: Michel Beaudouin-Lafon}{}{ We implemented the interaction technique -- \Colorhref{https://engineering.purdue.edu/~elm/projects/polyzoom/polyzoom.pdf}{PolyZoom} from CHI 2012 by building a mash-up application which uses the technique. The application allows user to navigate tweets around the world by progressively building hierarchies of focus regions (stacked on top) with certain magnification. The hierarchies can be created and compared side by side which provides spatial context for tweets and therefore helps in comparing tweets from spatially distant parts of the world simultaneously. The applications is built on nodeJS, by using APIs such as Twitter search API and Google maps API.
}


\cvitem{Project} {\textbf{\textsf{Semester Project: Computer Vision Labratory}}}
	\cventry{\scriptsize{Oct 2011--Nov 2011}}{\textnormal {\em{Student, CourseProject}}}{ETH Z\"urich}{Advisers: Dr. Helmut Grabner, Prof. Luc Van Gool}{}{ I worked on a project to improve the machine learning algorithm that detects and categorizes objects based on physical affordances. We  investigated various parameters like physical stability of the objects, material properties, etc through which the algorithm could be improved for better classification results in a 3d scene.\newline
}



\cvitem{Project} {\textbf{\textsf{libDB \& libSpell: Libraries with API for relational database access  and cross platform spelling check}}}
\cventry{\scriptsize{Feb 2011--May 2011}}{\textnormal {\em{Student, CourseProject}}}{ETH Z\"urich}{Professor: Bertrand Meyer}{}{ We designed and implemented two libraries which provided APIs to help programmers access RDBs from Eiffel and provide spelling hints during programming. The libSpell library uses google toolbar api in the backend. Throught the project, we followed various methods of software development and designed the API with concepts we learned during the course such as principles of software architecture, design patterns, design by contracts, etc. \newline{}%
}



\section{Languages}
\cvitemwithcomment{English}{Fluent}{}
\cvitemwithcomment{German}{Intermediate}{}
\cvitemwithcomment{Tamil}{Native}{}
\cvitemwithcomment{French}{Basic}{}



\section{References}
\textbf{Yvonne Jansen}, Postdoctoral researcher, Department of Computer Science, University of Copenhagen, Denmark. email: jansen@lri.fr\\\\
\textbf{Pierre Dragicevic}, Research Scientist, INRIA, France.\\email: dragice@lri.fr\\\\
\textbf{Moira Norrie}, Professor, Global Information Systems Group, {ETH Z\"urich}, {Z\"urich}.\\
email: norrie@inf.ethz.ch\\


\textbf{Michael Nebeling}, Postdoctoral researcher \& Lecturer, Global Information Systems Group, {ETH Z\"urich}, {Z\"urich}. \\
email: nebeling@inf.ethz.ch














%\clearpage\end{CJK*}                         % if you are typesetting your resume in Chinese using CJK; the \clearpage is required for fancyhdr to work correctly with CJK, though it kills the page numbering by making \lastpage undefined
\end{document}


%% end of file `template.tex'.


